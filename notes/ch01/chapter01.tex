\providecommand{\main}{../..}%define path to bib for subfiles
\documentclass[../../main.tex]{subfiles}
\begin{document}
\onlyinsubfile{
\setcounter{chapter}{8}
}
\notinsubfile{}
\chapter{quantum country}
\onlyinsubfile{
\marginpar{\vspace{0cm}
\textcolor{red}{ hoofdstuk is los gecompileerd, hstk nummer is \thechapter}
}
}
\notinsubfile{
\marginpar{\vspace{0cm}
\textcolor{red}{main.tex gecompileerd, nummering zou moeten kloppen}}
}
\section*{aantekeningen uit quantum country}

\subsection*{Quantum Computing for the Very Curious}

Q: write the following in ket notation\\
$\begin{pmatrix}
\sqrt{0.7}\\
\sqrt{0.3}
\end{pmatrix}
$

$\sqrt{0.7} \ket{0}+\sqrt{0.3}\ket{1}$


Q:It is useful to think of the left to right of a quantum wire as a passage of ...
\\
A:time

Q:What is the matrix representation of the X-gate\\
A:
$X \equiv \begin{pmatrix}
0&1\\
1&0
\end{pmatrix}
$

Q:How many computational basis states does a qubit have?\\
A:2

Q:How an you compute the length of
$H\begin{Vmatrix}
1\\
3
\end{Vmatrix}
$
without explicitely computing the product of the Hadamard gate and the vector?
What is the length?\\

A:Hadamard operator leaves the length unchanged. The length is $\sqrt{1^2+3^2}=\sqrt{10}$

Q:Suppose we have a qubit in the state
$$\frac{\ket{0}+\ket{1}}{\sqrt{2}}$$. What is the probabilitiy of a measurement in the in the computational basis will give the result 0? What is then its posterior state? 

A:\SI{50}{\percent}, $\ket{0}$

Q:What are three types of physical systemes that potentially can be used to store qubits?\\

A:photons, electrons, atoms, doted diamonds

Q:What do we call the two-dimensional vector space where the state of a qubit lives?\\
A:State space

Q:What is the inverse of a Hadamard gate?\\
A:Hadamard gate

Q:What does the Hadamard gate do to the state $\ket{0}$?\\
A:$\ket{+}$

Q:After we measure $\alpha\ket{0}+\beta\ket{1}$ in the computational base, is it still in the state  $\alpha\ket{0}+\beta\ket{1}$?\\
A:No, it's posterior state is either  $\ket{0}$ ot  $\ket{1}$

Q:How can the following circuit be simplified?
\begin{center}
\leavevmode
\Qcircuit @C=1em @R=2em {
& \gate{H} & \gate{H} & \qw  \\
}
\end{center}
A:
\begin{center}
\leavevmode
\Qcircuit @C=1em @R=2em {
& \qw & \qw & \qw  \\
}
\end{center}

Q: What is the result of the operator product XX?\\
A:I

Q:How many dimensions does the state vector of a qubit have?\\
A:2

Q:How can the following circuit be written?
\begin{center}
\leavevmode
\Qcircuit @C=1em @R=2em {
\lstick{\ket{\Psi}} & \gate{X} & \gate{H} & \qw  \\
}
\end{center}
$HX\ket{\Psi}$ or $XH\ket{\Psi}$\\

A:$HX\ket{\Psi}$

Q:Is $\alpha\ket{0}+\beta\ket{1}$ the same as $\beta\ket{1} + \alpha\ket{0}$?Why?\\
A:same, order does not matter in plus operator

Q:How large is the matrix representing a single qubit gate?\\
A:2x2

Q:What is a quantum gate that can distinguish $\ket{+}$ and $\ket{-}$?\\
A:H

Q:Suppose we have a bit in the quantum state $\ket{1}$. What is the probability a measurement in the computational basis gives the result 0?  and 1?\\
A:0, 1


Q:A superposition of quantum states is the same as a .. of quntum states\\
A: linear combination

Q:What are thee Pauli matrices:\\
A:XYZ (and I)


Q:What is $X\ket{0}$\\
A:$\ket{1}$

Q:What does the normalisation condition for the quantum states mean for the probabilities for a measurement in the computational basis?\\
A: They add up to 1

Q:Suppose we measure $\ket{+}$ or $\ket{-}$. Are the probability distributions of the outcome the same or different for these qubits?\\
A:The same

Q:What is a geometric interpretation of U being a unitary operation?\\
A:length preserving, e.g. rotation or reflection.

Q:What is the second quantum gate Alice applies to her qubits in the teleportation protocol?
A:A Hadamard gate to the first qubit i.e. the one that started as $\ket{\psi}$

Q:Where does the term classical in the term 'classical bits' come from?

A:classical physics

Q:Suppose we have a quantum state $\sqrt{0.8}\ket{01}+\sqrt{.2}\ket{10}$ and we measure the first qubit in the computational basis. Supposing we get 0 as the outcome, what is the corresponding state of the second qubit?

A:$\ket{1}$ with \SI{100}{\percent} certainty

Q:What is the amplitude of the $\ket{1}$ state in the ket 
$\sqrt{0.7}\ket{0}+\sqrt{0.3}\ket{1}$

A:$\sqrt{0.3}$

Q:What is a quantum circuit showing the X gate being applied to a single qubit?

A:
\begin{center}
\leavevmode
\Qcircuit @C=1em @R=2em {
& \gate{X} & \qw  \\
}
\end{center}

Q:Suppose we have a quantum state $\sqrt{0.8}\ket{01}+\sqrt{.2}\ket{10}$ and measure the first qubit in the computational basos. Supposing we get one as an outcome, what is hte corresponding state of the second qubit?\\
A:$\ket{0}$ with \SI{100}{\percent} certainty

Q:How do the probabilities for the measurement outcomes in the teleportation protocol depend upon the amplitudes $\alpha$ an $\beta$ in the state $\alpha\ket{0}+\beta\ket{1}$ being teleported?\\
A:They are independent. They do not dependent of $\alpha$ and $\beta$ at all.

Q:We use the term ket interchangebly with the term\\
A:(column) vector

Q:Why can't quantum teleportation be used to transmit a quantumstate $\ket{\Psi}$ faster than light?\\
A:To teleport $\ket{\psi}$ to Bob, Alice must send Bob two bits of classical information. The tranmission of classical information is limited by the speed of light.

Q:How are the two computational basis states of a qubit usually written?\\
A:$\ket{0}$ and $\ket{1}$

Q:How does the X gate act on a general state of a qubit?\\
A:Interchanges the coefficients

Q:Why is it that systems which make good quantum wires are often hard to build quantum gates for?\\
A:Systems which make good quantum wires interact weakly with other systems; to do a quantum gate we need to manipulate the qubit, and it is hard to manipulate a system which only weakly interacts with oter systems.

Q:how many qubits are involved in a teleportation protocol?\\A:3

Q:Why do we rather write $X\ket{0}$ rather than $X(\ket{0})$?\\
A:It is a matrix operation on a vector rather than a function applied on an argument.

Q:Why would a neurtrino make a good quantum wire?\\
A:Neutrinos interact very weakly with other matter, which could make it very stable.

Q;How many qubits are directly involved in Alices's part of the teleportation protocol?\\
A:2

Q:How many qubits are directly involved in Bob's part of the telepostation protocol?\\
A:1

Q:Why are quantum wires often hard to implement?\\
A:Because qubits are fragile, and their state can easily be disturbed.

Q:In the quantum teleportation protocol, what is the 2-bit state initially shared between Alice and Bob?\\
A: $\frac{\ket{00}+\ket{11}}{\sqrt{2}}$

Q:What is the quantum circuit notation for a quantum wire?\\
A:\begin{center}
\leavevmode
\Qcircuit @C=1em @R=2em {
& \qw & \qw & \qw  \\
}
\end{center}

Q:What do we call the two-dimenssional vector space where the state of a qubit lives?\\
A:state space

Q:What is the result of applying the X-gate to the quantum state $$\frac{\ket{0}-\ket{1}}{\sqrt{2}}$$

A:$X\frac{\ket{0}-\ket{1}}{\sqrt{2}}= \frac{\ket{1}-\ket{0}}{\sqrt{2}}$

Q:How do the probabilities for the measurement outcomes in the teleportation protocol depend upon the amplitudes $\alpha$ and $\beta$ in the state $\alpha\ket{0}+\beta\ket{1}$ being teleported?

A:They are independent - they do not depend on those amplitudes at all.

Q:What is the total quantum circuit that Alice applies in a teleportation protocol?

A:\\
\vspace{1cm}
\Qcircuit @C=1em @R=2em {
& \ctrl{1} & \gate{H}    & \gate{Z} & \cw \\
& \targ    & \qw        & \gate{X}  & \cw\\
}

Q:We could rewrite the sequence:
\hfill
\Qcircuit @C=1em @R=2em {
& \lstick{\ket{\Psi}} & \gate{X}    & \gate{H} & \cw \\
}
\hfill
as..\\
A: $HX\ket{\Psi}$

Q:After a measurement, is a qubit in the state 
$\alpha\ket{0}+\beta\ket{1}$
 still in that state?

A:No

Q:Suppose we have a qubit in the state $\frac{\ket{0}+\ket{1}}{\sqrt{2}}$. What is the probability that a measurement in the computational basis will give the result 0? What is the posterior state if that outcome occurs?

\SI{50}{\percent}, $\ket{0}$ 

Q:Suppose we have the state $\ket{0}$. What is the probability a measurement in the computational base gives the result 0? What is the probability the measurement gives the result 1?

A:\SI{100}{\percent}, 1

Q:Suppose we have a qubit in the state $$\frac{\ket{0}+\ket{1}}{\sqrt{2}}$$. what is the probabilitty that a measurement in the computational basis will give the result 1? What is the posterior state if that outcome occurs?

A:\SI{50}{\percent}, $\ket{1}$

Q:Suppose we have the state $\ket{1}$. What is the probability a measurement in the computational base gives the result 0? What is the probability the measurement gives the result 1?

A:0, 1

Q:How do we denote a computational basis measurement in the circuit model?\\
A:
\begin{center}
\leavevmode
\Qcircuit @C=1em @R=2em {
&\qw &  \meter &\cw \\
}
\end{center}

Q:Suppose we do computational basis measurements for either $\ket{+}$ or $\ket{+}$. Are the probability distributions for outcomes the same or different for these states?\\
A:The same

Q:Suppose you had a cdevice that could exactly determine the state of a qubit. How could such a device be used as part of a scheme to communicate infinite classical information, using a single qubit.

A:The sender of the classical bits could encode the (infinite string of) bits in the binary expansion of the real part of the amplitude $\alpha$ in the state 
$\alpha\ket{0}+\beta\ket{1}$.
The receiver of the qubit could figure out $\alpha$ and thus the entire string of bits.

Q:What does the normalization condition for quantum states mean about the probabilities for a measurement in the computational basis?

A:Their probabilities for measurement outcomes sum up to 1.


Q:What is a quantum circuit with which we can distinguish the states $\ket{+}$ and $\ket{-}$?

H -> measurement
\Qcircuit @C=1em @R=2em {
 & \gate{H}    & \meter & \cw \\
}


\subsubsection*{General single-qubit gates}


$U^{\dagger} \equiv (U^T)^* $

$\begin{pmatrix}
a&b\\
c&d
\end{pmatrix}^*
=
\begin{pmatrix}
a^*&b^*\\
c^*&d^*
\end{pmatrix}
$


$Z \equiv \begin{pmatrix}
1&0\\
0&-1
\end{pmatrix}
$

ASo Z leaves $\ket{0}$ unchanged and maps $\ket{1}$ to $-\ket{1}$
Show that X, H, I, Z are unitary



we get into complex numbers, skip for course.

$Y \equiv \begin{pmatrix}
0&-i\\
i&0
\end{pmatrix}
$

rotation

$Y \equiv \begin{pmatrix}
cos \theta & -sin \theta\\
sin \theta & cos \theta
\end{pmatrix}
$


Q:A single qubit is represented as a 2x2 ... matrix\\
A:Unitary

Q:what is $H^\dagger$?
A:$H$

Q:What is the algebraic condition defining unitarity for a matrix U?\\
A: $UU^\dagger=I$

Q:How large is the matrix repersenting a single qubit-state?\\
A:2x2

Q:What are three common names for the dagger operation?\\
A:
adjoint operation,
dagger operation,
transpose complex conjugate,
Hermitian conjugate operation

Q:What notation we use to denote Pauli matrices?\\
A:I, X, H, Z

Q:What is the element in the bottom left corner of a Y-gate?
A:

\subsubsection*{What does it mean if a matrix is unitary?}

Unitary operators preserve the length of their inputs. (cf. rotations, reflections).

This means that once normalised, as quantum gates do not change length, the outcome is normalised.

Unitary matrices are the only matrices that preserve length, they are the exact class that preserve length.

Holds also for N dimensions

Proof ...

how can you compute the legtht of 



$H\begin{Vmatrix}
3 \\
10 
\end{Vmatrix}
$

$\sqrt{10}$
 
important 
 $\|U\ket{\psi}\| = \|\ket{\psi}\|$
 
\subsubsection*{Why are unitaries the only matrices which preserve length?}
 
 
$$(M\ket{\psi})^\dagger = \bra{\psi}M^\dagger$$ 


$$\|M\ket{\psi}\|^2 = \bra{\psi}M^\dagger M \ket{\psi}$$


$M\ket{e_j}$ is the kth column of M and that $\bra{e_j}M\ket{e_k}$ is the jkth element of M.





\subsubsection*{summary of teleportation protocol}

Q:What is the starting state of for the teleportation protocol?
A:$\ket{000}$

\begin{enumerate}
\item Initial state: Alice has a qubit $\ket{\psi}$. Alice and Bob each prepare a bit in the $\ket{0}$ state, entangle these through a H and CNOT gate and each take one qubit of this now entangled pair.
\item Alice apllies a CNOT between $\ket{\psi}$ and hetubit, applies a H-gate to the first bit of the outcome. She measures both her qubits in the computational basis getting results z=0 or 1 and x=0 or 1. The probability of each othe ooutcomes (00, 01, 10, 11) is $\tfrac{1}{4}$. 
\item Classical communication: Alice broadasts both classical bits z and x
\item Bob recovers the quantum state of $\ket{\psi}$. Bob applies $Z^z$ and $X^x$ and recovers $\ket{\psi}$.
\end{enumerate}

Alice no longer possesses $\ket{\psi}$. It is teleoprted not copied!.

review questions

What is the first gate Alice applies to her qubits in the teleportation protocol

CNOT with $\ket{\psi}$ as the control and the other as target

How many classical bits Alice has to send to Bob in the teleportation protocol?

2

To recover the teleported state Bob applies combinations of the Pauli .. and .. matrices

X and Z


Dit circuit moet je eens uitschrijven.
\begin{center}  %DE manier om figuur te ontfloaten. Gebruik package caption in de preambule
\leavevmode
\Qcircuit @C=1em @R=2em {
\lstick{\ket{+}}  & \qw  & \ctrl{1}  & \qw    & \qw  & \rstick{\ket{-}}\\
\lstick{\ket{-}}  & \qw  & \targ     & \qw    & \qw  & \rstick{\ket{-}}
}
\end{center}

$$CNOT\ket{C,T}=CNOT\ket{+,-}$$

$CNOT = 
\begin{pmatrix}
1&0&0&0\\
0&1&0&0\\
0&0&0&1\\
0&0&1&0\\
\end{pmatrix}
$
,
$\ket{+}=
\begin{pmatrix}
\tfrac{1}{\sqrt{2}}\\
\tfrac{1}{\sqrt{2}}
\end{pmatrix}
$
,
$\ket{-}=
\begin{pmatrix}
\tfrac{1}{\sqrt{2}}\\
\tfrac{-1}{\sqrt{2}}
\end{pmatrix}
$


$
CNOT\ket{+-}=CNOT
\begin{pmatrix}
\begin{pmatrix}
\tfrac{1}{\sqrt{2}}\\
\tfrac{1}{\sqrt{2}}
\end{pmatrix}
\otimes
\begin{pmatrix}
\tfrac{1}{\sqrt{2}}\\
\tfrac{-1}{\sqrt{2}}
\end{pmatrix}
\end{pmatrix}
=
\frac{1}{2}
\begin{pmatrix}
1&0&0&0\\
0&1&0&0\\
0&0&0&1\\
0&0&1&0\\
\end{pmatrix}
\begin{pmatrix}
1\\
-1\\
1\\
-1
\end{pmatrix}
=
\frac{1}{2}
\begin{pmatrix}
1\\
-1\\
-1\\
1
\end{pmatrix}
=
\frac{1}{2}
\begin{pmatrix}
1\\
-1
\end{pmatrix}
\otimes
\begin{pmatrix}
1\\
-1
\end{pmatrix}
=
\begin{pmatrix}
\tfrac{1}{\sqrt{2}}\\
\tfrac{-1}{\sqrt{2}}
\end{pmatrix}
\otimes
\begin{pmatrix}
\tfrac{1}{\sqrt{2}}\\
\tfrac{-1}{\sqrt{2}}
\end{pmatrix}
=
\ket{-,-}
$

NB in Quantum inspire levert dit als antwoord de hele computational basis op met gelijke kansen. Ik kan daar geen $\ket{-}$ terugvinden

Nou dat is wat! In de Hadamard basis (hoe heet deze pendant van de computational basis) is de toestand van de control bit niet behouden!

Nu is de inverse van een CNOT natuurlijk de CNOT. Wat betekent dat voor het bovenstaannde circuit?
[is ook van rechts naar links te lezen]

\subsection*{quantum country, QC for the very curious part 3}
Een qc programma ziet er altijd zo uit:
\begin{itemize}
\item Start in een gedefinieerde basis meestal computational basis (cb), z-richting.
\item pas een aantal CNOT en single qubit operatoren toe
\item eindig met een meting in de cb
\end{itemize}

Q:What is the initital step of a quantum computation?
A:preparation (most often in the cb)

Q:What is the last step in a qcomputation?
A: ameasurement in the cb

Q: is the product of two unitary matrices also unitary
A: Yes (apply them one by one and you'll see)

Q: Suppose I introduceed a gate $e^{i\theta}I$. Does it affect the outcome of a circuit?
A: N, thre resulting phase shift does not influece the amplitudes of the coefficients of the state vectors. The amplitues determine the measurement.

Q: the X-operation and the -X operation are the same except a ... factor.
A: global phase factor



A toffoli gate can be be used to model a AND gate.
\begin{center}  %DE manier om figuur te ontfloaten. Gebruik package caption in de preambule
\leavevmode
\Qcircuit @C=1em @R=2em {
\lstick{\ket{x}}  & \qw  & \ctrl{2}  & \qw    & \qw  & \rstick{\ket{x}}\\
\lstick{\ket{y}}  & \qw  & \ctrl{1}  & \qw    & \qw  & \rstick{\ket{y}}\\
\lstick{\ket{z}}  & \qw  & \targ     & \qw    & \qw  & \rstick{\ket{z\oplus (x \wedge y)}}
}
\end{center}

exercise: build a NAND gate. A NAND is the nagteion of an AND gate

Solution:
\begin{center}  %DE manier om figuur te ontfloaten. Gebruik package caption in de preambule
\leavevmode
\Qcircuit @C=1em @R=2em {
\lstick{\ket{x}}  & \qw & \gate{X}  & \ctrl{2}  & \qw    & \qw  & \rstick{\ket{x}}\\
\lstick{\ket{y}}  & \qw & \gate{X}  & \ctrl{1}  & \qw    & \qw  & \rstick{\ket{y}}\\
\lstick{\ket{z}}  & \qw & \qw & \targ     & \qw    & \qw  & \rstick{\ket{z \oplus (x \wedge y)}}
}
\end{center}

Er is nog een simpeler oplossing:

Solution:
\begin{center}  %DE manier om figuur te ontfloaten. Gebruik package caption in de preambule
\leavevmode
\Qcircuit @C=1em @R=2em {
\lstick{\ket{x}}  & \qw  & \ctrl{2}  & \qw    & \qw  & \rstick{\ket{x}}\\
\lstick{\ket{y}}  & \qw  & \ctrl{1}  & \qw    & \qw  & \rstick{\ket{y}}\\
\lstick{\ket{1}}  & \qw  & \targ     & \qw    & \qw  & \rstick{\ket{1 \oplus (x \wedge y)}}
}
\end{center}


Q:What is the problem of storing athe pmplitudes of a many quantum system in a classical computer?
A:The number of amplitudes increases quickly with teh amount of qubits, 

Q: what numbertheoretical problem quantum computers appear to be good in solving?
A: factoring mumbers into prime factors.

Q:What is the name of the quantum algorithm to find primes factors of a number?
A" Shor's algorhythm

Q: Can quantum computers simulate the standard model or quantum gravity?
A: unknown (2x)

Q:There were successses in simulation QFT by ..
A; John proescill and colleagues



Excercise:show that the toffoli gate is reversible
Solution: The x, y give no problem, they are just copied ( in the cb). If the z is unchanged, its reverse will not change it eiter, since y and y are unchanged. If it it is chaged, the repetitive application will change it back. resulting in the I-operator in all cases where the toffoli has been applied twice.
In matrix operation the problem is only in the lower richt corner.squaring this matrix yields identity.

\subsection*{quantum country, how quantum teleportation works}


Q:What quantum circuit prepares a 
$$\frac{\ket{00}+\ket{11}}{\sqrt{2}}$$
 state at the start of a teleportation?

A: (0,0) H CNOT

Q:If Bob prepares the state
$$\frac{\ket{00}+\ket{11}}{\sqrt{2}}$$, shared at the start of the teleportation protocol, why does it not matter which qubit he sends to Alice?

A:Because the states are symmetrical.

 
Q:Is it possible to use quantum teleportation to transmit information faster than light?

A:No


\subsection*{How quantum teleprotation works}
\subsubsection*{The teleprotation protocol}
\subsubsection*{How to remember the teleprotation protocol}
\subsubsection*{Does teleportation protocol allow faster than light communication?}
\subsubsection*{How partial measurements work}


Q:What quantum circuit prepares a 
$$\frac{\ket{00}+\ket{11}}{\sqrt{2}}$$
 state at the start of a teleportation?

A: (0,0) H CNOT
 
If Bob prepares the state
$$\frac{\ket{00}+\ket{11}}{\sqrt{2}}$$, shared at the start of the teleportation protocol, why does it not matter which qubit he sends to Alice?
A:Because the states are symmetrical.
 
Q:Is it possible to use quantum teleportation to transmit information faster than light?

A:No






\subsubsection*{How partial measurements work}

Suppose we measure the first two qubits of a thre bit system in the computational baiss. Wat are the possible states of the outcome?

$\ket{00}$,$\ket{10}$,$\ket{01}$,$\ket{11}$


Q:Suppose we do a measurement in teh cb of the first two bits of a three pit system. Wat are the possible outcomes?

A:00, 01, 10, 11
%A:$\ket{00}$,$\ket{10}$,$\ket{01}$,$\ket{11}$


Q:Suppose we have a quantum state 
$\sqrt{0.8}\ket{0}+\sqrt{0.2}\ket{1}$
and measure the first qubit in the computational baiss. What is the probability the measurement gives 1 as an outcome

A:0.2

Q:Suppose we have a quantum state 
$\sqrt{0.8}\ket{0}+\sqrt{0.2}\ket{1}$
and measure the first qubit in the computational baiss. What is the probability the measurement gives 0 as an outcome

A:0.8




\subsubsection*{Verifying that the teleportaion protocol works}

$$(\alpha\ket{0}+\beta\ket{1})\frac{\ket{00}+\ket{11}}{\sqrt{2}}$$

expand:


$$\frac{\alpha\ket{000}+\alpha\ket{011}+\beta\ket{100}+\beta\ket{111}}{\sqrt{2}}$$

Apply CNOT to the first two qubits:

$$\frac{\alpha\ket{000}+\alpha\ket{011}+\beta\ket{110}+\beta\ket{101}}{\sqrt{2}}$$


Now we apply a Hadamard to the first qubit

$$\frac{\alpha\ket{000}+\alpha\ket{100}+
\alpha\ket{011}+\alpha\ket{111}+
 \beta\ket{010}- \beta\ket{110}+
 \beta\ket{001}- \beta\ket{101}}
{2}$$

$$\frac{
\ket{00}(\alpha\ket{0}+\beta\ket{1})+
\ket{01}(\alpha\ket{1}+\beta\ket{0})+
\ket{10}(\alpha\ket{0}-\beta\ket{1})+
\ket{11}(\alpha\ket{1}-\beta\ket{0})+
}{2}$$

When alice meeasures in the computational base, the outcome is $\ket{00}$ with probability given by $\frac{\alpha^2+\beta^2}{4}=\tfrac{1}{4}$

The resulting state for Bob is $\alpha\ket{0}+\beta\ket{1}$

The results for the other outcomes of the measurements in Alice's computational basis are:



\begin{table}[]
\centering
\begin{tabular}{l|l|l}
\cline{1-3}
Outcome   & Probability   & Bob's state   \\ \cline{1-3}
 00   &  $\tfrac{1}{4}$    & $\alpha\ket{0}+\beta\ket{1}= \ket{\psi}$     \\ %\cline{1-3}
 01   &  $\tfrac{1}{4}$    & $\alpha\ket{1}+\beta\ket{0}= X\ket{\psi}$     \\% \cline{1-3}
 10   &  $\tfrac{1}{4}$    & $\alpha\ket{0}-\beta\ket{1}= Z\ket{\psi}$     \\ %\cline{1-3}
 11   &  $\tfrac{1}{4}$    & $\alpha\ket{1}-\beta\ket{0}= XZ\ket{\psi}$     \\% \cline{1-3}
\end{tabular}
\caption{metingen}
\label{metingen}
\end{table}



Now Bob's state is very similar to the original $\ket{\psi}$.
He is only a few Pauli gates off. He has to do either nothing, apply X, apply Z or apply ZX respectively.
The protocol he has apply is encoded in the calssical bits.


Here a copy of the circuit
\begin{center}  %DE manier om figuur te ontfloaten. Gebruik package caption in de preambule
\leavevmode
\Qcircuit @C=1em @R=2em {
\lstick{T} & \ustick{\ket{\Psi}} & \qw     & \qw       & \targ     & \gate{H}   & \qw      & \meter \cwx[2] \\
\lstick{A} & \ustick{\ket{0}}    & \gate{H}& \targ     & \ctrl{-1} & \qw        & \meter \cwx[1]  \\
\lstick{B} & \ustick{\ket{0}}    & \qw     & \ctrl{-1} & \qw       & \qw        & \gate{X} & \gate{Z} & \qw & \ustick{\ket{\Psi}}
}
\end{center}


Note1: The calssical bits reveal nothing about state $\ket{\psi}$. 
Note2: If Eve steals the classical ibits she cannot retrieve $\ket{\psi}$.

The measurements are saying how the states are changed to $\ket{\psi}$, $\ket{X \psi}$, $\ket{Z \psi}$, and $XZ \ket{\psi}$. without giving any information on $\ket{\psi}$!

Note3: It does not matter where bBob is in this story. Alice may broadcast the classical bits over the internet.

What are the probabilities for the outcomes of the teleportation protocol?

$\tfrac{1}{4}$ for all outcomes

Suppose Alice doesn't know where bob is. How can she transmit the two classical bits so Bob can complete the teleortation protocol.

\subsubsection*{Summary of the teleportation protocol}
staat hierboven 

Q:what is the starting state for a teleportation protocol?

A:$$\ket{\\psi}\frac{\ket{00}+\ket{11}}{\sqrt{2}}$$

Q:How many classical bits does Alice send to Bob in the teleportation protocol?

A:2

Q:To recover the teleportated state, Bob applies {auli gate .. and ..

A: X, Z

Q:What is the first qgate Alice applies to her qubits in the teleportation protocol?

A:CNOT with $\ket{\Psi}$ as control and one of the the entangled qubits as target 

Q:



\section*{aantekeningen uit EdEx course}
Which statements are true for physical qubits
\begin{itemize}
\item +physical qubits can be in superposition
\item physical qubits consist of one or more physicsl qubits
\item physical qubits have longer coherence time tha physical qubits
\item +physical qubits are pysically realised qubites
\item physical qubits are three level qubits
\end{itemize}

Which statements are true for logical qubits
\begin{itemize}
\item +logical qubits can be in superposition
\item +logical qubits consist of one or more physicsl qubits
\item +logical qubits have longer coherence time tha physical qubits
\item logical quybits are pysically realised qubites
\item logical qubits are three level qubits
\end{itemize}

One important fact about comparing classical and quantum computers is that everything a quantum computer can do a classical computer can also do. However, a lot of classical computer memory is needed to simulate a small number of qubits.

To simulate $n$ qubits on a classical computer, you need $64 \times 2^n$ classical bits (working with double precision floating point numbers).

There are classical computers that have access to about 5000 terabytes ($500 \times 10^{14} \mathrm{bits}$) of memory. How many logical bits are needed to perform a computation that cannot be simulated on a classical computer?
\begin{itemize}
\item about 5
\item =about 50
\item about 500
\item about 5000
\end{itemize}

When evaluating a design for a new quantum computer, it is helpful to have a checklist of known requirements that the computer has to fullfil. The DeVincenzo cirteria provide such a checklist.

Which five of the following statements are DiVincenzo criteria for the implementation of quantum computation?
\begin{itemize}[nosep]
\item A quantum system must have the ability to work at room temperature
\item +Good qubits are needed, the quantum state cannot be lost
\item +A quantum system must have a universal set of quantum gates (a universal set of gates is a set such that any of the possible gates can be rewritten in a sequence of the gates in the universal set).
\item It must be possible to send qubits between two locations.
\item +A quantum system must be scalable, with well-defined qubits
\item +It must be possible to perform measurements on a qubits in a quantum system
\item A quantum system must have the ability to be faster than a classical computer
\item +A quantum system must be able to initialize qubits to a fixed state, such as the zero state.
\item It must be possible to transfer stionary qubuits (qubits on a quantum chip) to flying qubits (qubits that are sent to another location over a quantum channel and back). 
\end{itemize}

flimpje: Main takaways:
\begin{itemize}
\item A quantum internet is composed of end nodes, witches, repeaters and control traffic.
\item An entangled state between two qubits is the essence of the poer of a qunautm internet.
\item Qubits can be entagled at a very long distance, but when we make the same measurement on both qubits, they will give the same outcome. This feature is called maximum coordination.
\item when two qubits are maximally entangled, it is impossible for any other qubit to have a share of this entanglement, making it inherently private.
\item 
\end{itemize}

Quantum versus classical internet

In the lecture, some aspects of a quantum internet were introduced. You will compare these with those of a classical internet.

Which of the following features are possible on a quantum internet, but not on a classical internet?

\begin{itemize}
\item +Secure communication
\item Faster than light communication
\item Anonymous communication
\item +Secure cloud computing
\end{itemize}

Size of an end node

In the lecture Stefanie discussed the end nodes of a quantum internet, and said that a large quantum computer is not necessary in order to have a functioning end node.

Why can the quantum computers at the end nodes be very small?
\begin{itemize}
\item Because the number of nodes is very small
\item =Because one qubit at each end node is enough to have entanglement over the whole network.
\item Because we can use repeaters to increase the number of qubits at each node.
\end{itemize}

In the video, Stefanie explained features of the entanglement between two qubits.

Which of the following statements are true about the entanglement of qubits?
\begin{itemize}
\item =Measuring one qubit of a pair of maximally entangled qubits is is enough to know the state of the other qubit as well.
\item Only two qubits can be entangled.
\item Qubits can be copied.
\item =Maximal entanglement is inerently private.
\end{itemize}

5. The nature of entanglement

What is the definition of macimal coordination?

\begin{itemize}
\item Maximum coorination means that, if measured, one qubit sends a message to the other qubit, telling it to give the same output.
\item =Maximum coorination means that, if measured the same way, two entangled qubits always give the same outcome.
\item Maximum coorination means that one entangled qubit is a copy of the other.
\end{itemize}

Why is a quantum computer with few (i.e. about five) logical qubits also relevant?
\begin{itemize}
\item It can do some tasks exponentially faster than a classical computer.
\item =It is useful for scientists and quantum software engineers to test their concepts.
\item =It is already possible to securely communivate with just two end nodes, each with one qubit.
\item It is possible to solve some problems with five qubits that a classical computer cannot solve.
\end{itemize}


There are two more DiVincenzo criteria:
\begin{enumerate}\addtocounter{enumi}{5}
\item The ability to interconvert stationary and flying qubits
\item The ability to faithfullly transmit flying qubits between specified locations
\end{enumerate}

(info from Mennos lecture 'from one to many qubits')

\begin{itemize}
\item =A quantum internet requires flying qubits
\item A flying qubit is always a photon
\item Any qubit interacting with other qubits is a flying qubit
\item The ability to apply quantum gates to flying qubits is required for quantum communication.
\end{itemize}
Main takaways

\begin{itemize}
\item There is a great symmetry in multiplying and factoring: multiplying is easy, factoring is difficult.
\item This symmetry is udsed in cryptography
\item Shor's quantum factoring algorithm: an algorithm which makes factoring possible in reasonable time, somthing that is not possible on an classical computer.
\end{itemize}
Shor's algorithm brings the problem of factoring from exponential to polynomal.


Een van Marcel Vonks werkbladen
\url{https://www.nemosciencemuseum.nl/media/filer_public/12/3b/123bdffb-0ec0-4e97-a4fc-5dfd5801c28a/quantum_en_relativiteit_quantum_en_relativiteit_rekenen_met_elektronen_leerling.pdf}

De quirky simulator 
\url{https://www.quantum-quest.nl/quirky/QuirkyQuest4.html}

Weblessen (careljan shouten)
\url{https://webcolleges.uva.nl/Mediasite/Catalog/catalogs/default}

Quantum secure authentication (twente)
\url{https://nymus3d.nl/portfolio/project/quantum-secure-authentication}


Why Everything You Thought You Knew About Quantum Physics is Different - with Philip Ball
\url{https://www.youtube.com/watch?v=q7v5NtV8v6I}

Quantum mechanics is weird:
\begin{itemize}
\item Quantum objects can be both particles and waves (particle-wave duality)
\item Quantum objects can be at more than one state at the time (superposition)
\item You cannot know exactly two properties of an quantum object (Heisenberg uncertainty principle)

\item Quantum objects can affect one another at huge distances instantly (entanglement)
\item You can't measure anything without disturbintg it (qm is subjective)
\end{itemize}

quantum random walk Quantum random walks - an introductory overview
\url{https://arxiv.org/pdf/quant-ph/0303081.pdf}

Wolfram demonstrations
\url{https://demonstrations.wolfram.com/QuantumRandomWalk/}

\url{https://demonstrations.wolfram.com/QuantumTeleportation/}


\section*{Aantekeningen Annemarije}

opg 14\\
eerst met $\ket{0}$ als controle
$
\begin{pmatrix}
\alpha\\
\beta
\end{pmatrix}
=
\alpha\ket{0}+\beta\ket{1}
$
\vspace{1cm}
\\
$
\Qcircuit @C=1em @R=2em {
& \lstick{\ket{0}}                  &  \qw   & \ctrl{1} & \qw  & \qw    \\
& \lstick{\ket{\alpha}+\ket{\beta}} & \gate{H} & \targ    & \gate{H} & \qw   \\
}
$
\vspace{1cm}
\\
$
H\begin{pmatrix}
\alpha\\
\beta
\end{pmatrix}
=
\tfrac{1}{\sqrt{2}}
\begin{pmatrix}
1&1\\
1&-1\\
\end{pmatrix}
=
\tfrac{1}{\sqrt{2}}
\begin{pmatrix}
\alpha+\beta\\
\alpha-\beta
\end{pmatrix}
$

$
CNOT(\ket{0},H(\alpha+\beta))=CNOT
\begin{pmatrix}
\begin{pmatrix}
1\\
0
\end{pmatrix}
\otimes
H\begin{pmatrix}
\alpha\\
\beta
\end{pmatrix}
\end{pmatrix}
=
$

$
\frac{1}{\sqrt{2}}
\begin{pmatrix}
1&0&0&0\\
0&1&0&0\\
0&0&0&1\\
0&0&1&0\\
\end{pmatrix}
\begin{pmatrix}
\alpha+\beta\\
\alpha-\beta\\
0\\
0
\end{pmatrix}
=
\frac{1}{\sqrt{2}}
\begin{pmatrix}
\alpha+\beta\\
\alpha-\beta\\
0\\
0
\end{pmatrix}
=
\frac{1}{\sqrt{2}}
\begin{pmatrix}
1\\
0
\end{pmatrix}
\otimes
\begin{pmatrix}
\alpha+\beta\\
\alpha-\beta
\end{pmatrix}
$
Het tweede bit is het target. Hierop laten we nogmaals H los:

$
H\frac{1}{\sqrt{2}}
\begin{pmatrix}
\alpha+\beta\\
\alpha-\beta
\end{pmatrix}
=
\tfrac{1}{\sqrt{2}}
\tfrac{1}{\sqrt{2}}
\begin{pmatrix}
1&1\\
1&-1\\
\end{pmatrix}
\begin{pmatrix}
\alpha+\beta\\
\alpha+\beta
\end{pmatrix}
=
\tfrac{1}{2}
\begin{pmatrix}
\alpha+\beta+\alpha-\beta\\
\alpha+\beta-\alpha+\beta
\end{pmatrix}
=
\tfrac{1}{2}
\begin{pmatrix}
2\alpha\\
2\beta
\end{pmatrix}
=
\begin{pmatrix}
\alpha\\
\beta
\end{pmatrix}
$

Nu met met $\ket{1}$ als controle
$
\begin{pmatrix}
\alpha\\
\beta
\end{pmatrix}
=
\alpha\ket{0}+\beta\ket{1}
$
\vspace{1cm}
\\
$
\Qcircuit @C=1em @R=2em {
& \lstick{\ket{1}}                  &  \qw   & \ctrl{1} & \qw  & \qw    \\
& \lstick{\ket{\alpha}+\ket{\beta}} & \gate{H} & \targ    & \gate{H} & \qw   \\
}
$
\vspace{1cm}
\\
$
H\begin{pmatrix}
\alpha\\
\beta
\end{pmatrix}
=
\tfrac{1}{\sqrt{2}}
\begin{pmatrix}
1&1\\
1&-1\\
\end{pmatrix}
=
\tfrac{1}{\sqrt{2}}
\begin{pmatrix}
\alpha+\beta\\
\alpha-\beta
\end{pmatrix}
$

$
CNOT(\ket{1},H(\alpha+\beta))=CNOT
\begin{pmatrix}
\begin{pmatrix}
0\\
1
\end{pmatrix}
\otimes
H\begin{pmatrix}
\alpha\\
\beta
\end{pmatrix}
\end{pmatrix}
=
$

$
\frac{1}{\sqrt{2}}
\begin{pmatrix}
1&0&0&0\\
0&1&0&0\\
0&0&0&1\\
0&0&1&0\\
\end{pmatrix}
\begin{pmatrix}
0\\
0\\
\alpha+\beta\\
\alpha-\beta
\end{pmatrix}
=
\frac{1}{\sqrt{2}}
\begin{pmatrix}
0\\
0\\
\alpha-\beta\\
\alpha+\beta
\end{pmatrix}
=
\frac{1}{\sqrt{2}}
\begin{pmatrix}
0\\
1
\end{pmatrix}
\otimes
\begin{pmatrix}
\alpha-\beta\\
\alpha+\beta
\end{pmatrix}
$
Het tweede bit is het target. Hierop laten we nogmaals H los:

$
H\frac{1}{\sqrt{2}}
\begin{pmatrix}
\alpha-\beta\\
\alpha+\beta
\end{pmatrix}
=
\tfrac{1}{\sqrt{2}}
\tfrac{1}{\sqrt{2}}
\begin{pmatrix}
1&1\\
1&-1\\
\end{pmatrix}
\begin{pmatrix}
\alpha-\beta\\
\alpha+\beta
\end{pmatrix}
=
\tfrac{1}{2}
\begin{pmatrix}
\alpha-\beta+\alpha+\beta\\
\alpha-\beta-\alpha-\beta
\end{pmatrix}
=
\tfrac{1}{2}
\begin{pmatrix}
2\alpha\\
-2\beta
\end{pmatrix}
=
\begin{pmatrix}
\alpha\\
-\beta
\end{pmatrix}
$

Heb ik een rekenfout gemaakt?
De oplossing in kansdichtheid lijkt gelijk.



\end{document}



