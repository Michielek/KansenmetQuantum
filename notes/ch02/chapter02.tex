\providecommand{\main}{../..}%define path to bib for subfiles
\documentclass[../../main.tex]{subfiles}
\begin{document}
\onlyinsubfile{\setcounter{chapter}{8}}
\notinsubfile{}
\chapter{Aantekeningen Annemarije}

Ik zag twee spelletjes: 
\begin{itemize}[nosep]
\item boter kaas en eieren (pag 12 opg11)
\item circuit raden (opg5 pag 22)
\end{itemize}

Circuit raden hoort mooi bij het Deutsch oracle. (zie Microsoft aantekeningen). Hiervoor is verstrengeling nodig, maar het levert de simpelsete meting die je met een Q computer kunt doen en niet met een C computer
NB: Als je een heleboel van deze metingen combineert maak je een sterke Q computer. (Mooie obseratrie van Martin).


Wat meer in detail:

H1: QM
Toestand en superpositie, eigenwaarde observabelen operatoren normering meting
Vanuit klassiek al gauw naar quantum. goed.
De golffunctie is abstract, op zichzelf niet waarneembaar. Toestand is niet waarneembaar, alleen de eigenwaarden.

H2:lin. Algebra
Vectoren, matrices,inproduct en verwachtingswaarde

Hier staan goede opgaven in, maar biedt wsch. meer dan we nodig hebben.

Meerdere deeltjes en verstrengeling (zou ik naar achter halen)

Mooie opdrachten, zijn wel pittig
Stern Gerlach staat een beetje los. Meting leidt tot oplossen golffunctie, van superpositie tot realisatie
Pag 12 Opg 10Verstrengeling: Is een toestand decomposeerbaar Mooi
Pag12 opg 11 boter kaas en eieren.

Pag 14 klassieke poorten
Qubits in superpositie
Blochbol (blochcirkel)

Recept rekenen en tekenen kan verder uitgewerkt worden.
Quantumpoorten XZH
CNOT met twee bits, niet verstrengeld

Opg 14 heb ik uitgewerkt.zie onder

H4:Verstrengeling en teleportatie
Is een systeem decomposeerbaar, soms niet. Dan pas zijn twee qubit verstrengeld

Twee qubits zijn verstrengeld als ze niet decomposeerbaar zijn volgens tensorregels.
Goede opgaven

H5: hardware: ik heb die regels elders gezien. Onder dit kopje ook ruimte voor de diversse methoden en instituten waar die methoden worden gebruikt.



opg 14\\
eerst met $\ket{0}$ als controle
$
\begin{pmatrix}
\alpha\\
\beta
\end{pmatrix}
=
\alpha\ket{0}+\beta\ket{1}
$
\vspace{1cm}
\\
$$
\Qcircuit @C=1em @R=2em {
& \lstick{\ket{0}}                  &  \qw   & \ctrl{1} & \qw  & \qw    \\
& \lstick{\ket{\alpha}+\ket{\beta}} & \gate{H} & \targ    & \gate{H} & \qw   \\
}
$$

$
H\begin{pmatrix}
\alpha\\
\beta
\end{pmatrix}
=
\tfrac{1}{\sqrt{2}}
\begin{pmatrix}
1&1\\
1&-1\\
\end{pmatrix}
=
\tfrac{1}{\sqrt{2}}
\begin{pmatrix}
\alpha+\beta\\
\alpha-\beta
\end{pmatrix}
$

$
CNOT(\ket{0},H(\alpha+\beta))=CNOT
\begin{pmatrix}
\begin{pmatrix}
1\\
0
\end{pmatrix}
\otimes
H\begin{pmatrix}
\alpha\\
\beta
\end{pmatrix}
\end{pmatrix}
=
$

$
\frac{1}{\sqrt{2}}
\begin{pmatrix}
1&0&0&0\\
0&1&0&0\\
0&0&0&1\\
0&0&1&0\\
\end{pmatrix}
\begin{pmatrix}
\alpha+\beta\\
\alpha-\beta\\
0\\
0
\end{pmatrix}
=
\frac{1}{\sqrt{2}}
\begin{pmatrix}
\alpha+\beta\\
\alpha-\beta\\
0\\
0
\end{pmatrix}
=
\frac{1}{\sqrt{2}}
\begin{pmatrix}
1\\
0
\end{pmatrix}
\otimes
\begin{pmatrix}
\alpha+\beta\\
\alpha-\beta
\end{pmatrix}
$

\vspace{1cm}
Het tweede bit is het target. Hierop laten we nogmaals H los:
$
H\frac{1}{\sqrt{2}}
\begin{pmatrix}
\alpha+\beta\\
\alpha-\beta
\end{pmatrix}
=
\tfrac{1}{\sqrt{2}}
\tfrac{1}{\sqrt{2}}
\begin{pmatrix}
1&1\\
1&-1\\
\end{pmatrix}
\begin{pmatrix}
\alpha+\beta\\
\alpha+\beta
\end{pmatrix}
=
\tfrac{1}{2}
\begin{pmatrix}
\alpha+\beta+\alpha-\beta\\
\alpha+\beta-\alpha+\beta
\end{pmatrix}
=
\tfrac{1}{2}
\begin{pmatrix}
2\alpha\\
2\beta
\end{pmatrix}
=
\begin{pmatrix}
\alpha\\
\beta
\end{pmatrix}
$

Grappig terug bij af. Hadden we dit sneller kunnen zien?
Natuurlijk. een $\ket{0}$ als controle biet laat het target ongemoeid. Een H-gate twee maal toepassen levert de identiteit.
(immers unitair, reversibel)

Ma je ook het volgende doen?
Uitgaande van 
$
\begin{pmatrix}
\alpha\\
\beta
\end{pmatrix}
=
\alpha\ket{0}+\beta\ket{1}
$


Nu met met $\ket{1}$ als controle
$
\begin{pmatrix}
\alpha\\
\beta
\end{pmatrix}
=
\alpha\ket{0}+\beta\ket{1}
$
\vspace{1cm}
\\
$$
\Qcircuit @C=1em @R=2em {
& \lstick{\ket{1}}                  &  \qw   & \ctrl{1} & \qw  & \qw    \\
& \lstick{\ket{\alpha}+\ket{\beta}} & \gate{H} & \targ    & \gate{H} & \qw   \\
}
$$
\vspace{1cm}
\\
$
H\begin{pmatrix}
\alpha\\
\beta
\end{pmatrix}
=
\tfrac{1}{\sqrt{2}}
\begin{pmatrix}
1&1\\
1&-1\\
\end{pmatrix}
=
\tfrac{1}{\sqrt{2}}
\begin{pmatrix}
\alpha+\beta\\
\alpha-\beta
\end{pmatrix}
$

$
CNOT(\ket{1},H(\alpha+\beta))=CNOT
\begin{pmatrix}
\begin{pmatrix}
0\\
1
\end{pmatrix}
\otimes
H\begin{pmatrix}
\alpha\\
\beta
\end{pmatrix}
\end{pmatrix}
=
$

$
\frac{1}{\sqrt{2}}
\begin{pmatrix}
1&0&0&0\\
0&1&0&0\\
0&0&0&1\\
0&0&1&0\\
\end{pmatrix}
\begin{pmatrix}
0\\
0\\
\alpha+\beta\\
\alpha-\beta
\end{pmatrix}
=
\frac{1}{\sqrt{2}}
\begin{pmatrix}
0\\
0\\
\alpha-\beta\\
\alpha+\beta
\end{pmatrix}
=
\frac{1}{\sqrt{2}}
\begin{pmatrix}
0\\
1
\end{pmatrix}
\otimes
\begin{pmatrix}
\alpha-\beta\\
\alpha+\beta
\end{pmatrix}
$

\vspace{1cm}
Het tweede bit is het target. Hierop laten we nogmaals H los:
$
H\frac{1}{\sqrt{2}}
\begin{pmatrix}
\alpha-\beta\\
\alpha+\beta
\end{pmatrix}
=
\tfrac{1}{\sqrt{2}}
\tfrac{1}{\sqrt{2}}
\begin{pmatrix}
1&1\\
1&-1\\
\end{pmatrix}
\begin{pmatrix}
\alpha-\beta\\
\alpha+\beta
\end{pmatrix}
=
\tfrac{1}{2}
\begin{pmatrix}
\alpha-\beta+\alpha+\beta\\
\alpha-\beta-\alpha-\beta
\end{pmatrix}
=
\tfrac{1}{2}
\begin{pmatrix}
2\alpha\\
-2\beta
\end{pmatrix}
=
\begin{pmatrix}
\alpha\\
-\beta
\end{pmatrix}
$

Na veel gereken: De oplossing in amplitude is niet hetzelfde maar in kansdichtheid wel, of heb ik toch een rekenfout gemaakt?


Ik heb geprobeerd dit in quantum inspire te stoppen. Daarin krijg is steevast met 25\% gelijke kans op alle oplossingen:
$\ket{00},\ket{01},\ket{10},\ket{11}$. Begrijp t nog niet helemaal

\end{document}



