\providecommand{\main}{../..}%define path to bib for subfiles
\documentclass[../../main.tex]{subfiles}
\begin{document}
\onlyinsubfile{\setcounter{chapter}{8}}
\notinsubfile{}
\section*{Glossary}
\onlyinsubfile{
}
\notinsubfile{
}

\nogdoen{dit is maar een beginnetje. deze glossary moet geintegreerd worden met de tekst, package glossaries}

\begin{table}[h!]
\leavevmode
\begin{tabular}{|l|l|}
\hline
title & description  \\ \hline
Glossary & woordenlijst \\ \hline

Hidden variable theory & verborgen variabelen theorie\\
superpotistion & superposite\\
Entanglement & verstrengeling\\
reversible&\\
measurement&\\
Qubit & qubit\\
pure state & zuivere toestand\\
mixed state& gemengde toestand\\
zuivere toestand& \\
gemengde toestand&\\

Shor's algorhythm&\\
BB84 & Charles Bennett and Gilles Brassard schreven in 1984 het eerste quantumecnryptie protocol. Voor BB84 is geen verstrengelng nodig.\\

Deutsch Oracle & \\
bra-ket & Dirac notatie\\
Deutsch&\\
Deutsch-Jozsa&\\

LSB & Least significant bit \\
 & het minst wegende bit (meest rechtse cijfer in een\\
 & getal). Misleidend in een parallelle quantumcomputer\\

MSB & Most significant bit \\
& het zwaarst tellende bit (meest linkse cijfer in een getal)\\
register&\\
quantum wire&\\
CNOT&\\
control & controle\\

target & doel (bij een CNOT) \\
input & invoer \\

output & uitvoer\\
quantumparallelisme& kracht van een qc neemt exponentieel toe met het aantal registers. Dit is de kracht van qc, maar reken je niet rijk, zonder goed algoritme verdunt het antwoord exponentieel.\\
gate & poort\\

finite state machine & eindige toestands machine\\

Bell basis &\\
computational basis &standaard basis\\
Hadamard basis&diagonale basis\\
Bell state& zie pag opg\\
verstrengeld& twee bits zijn verstrengeld als hun toestand niet te ontbinden is in basis vectoren (zie Bell state)
\end{tabular}
\end{table}


\end{document}