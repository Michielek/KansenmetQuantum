\documentclass[../main.tex]{subfiles}
\begin{document}
%% my chapter content
\onlyinsubfile{this only appears if leerdoelen.tex is compiled (not when main.tex is compiled)}
\notinsubfile{this only appears if main.tex is compiled (not when leerdoelen.tex is compiled)}
%% more of my chapter 1 content
%% 
\nogdoen{dit is maar een beginnetje. deze glossary moet geintegreerd worden met de tekst, package glossaries}
\section*{Leerdoelen}

In deze sectie staan de leerdoelen met kleurindicatie voor de moeilijkheidsgraad gegeven.

Aan het eind van deze module kunnen/weten/beheersen leerlingen
\begin{itemize}
\item Vector voorstelling van twee toestandsystemen
\item Operaties als matrix bewerkingen op twee-vectoren (re\"ele ruimte)
\item verstrengeling dmv tensorvermenigvuldiging
\item Dirac notatie als verkorte voorstelling van de toestandvariabelen 
\item toestand, toestandvariabele 
\item diverse poorten irrreversibel (set en unset)
\item  en reversibel (NOT, X, H, Z, CNOT)
\item enekele poorten in sequentie kunnen toepassen als matrixbewerking
\item flowdiagram lezen en maken van een opeenvolging van poorten
 \item kans uit kansdichtheidscoofficenten berekenen en andersom.
\item het verschil tussen reversibele en reversibele operaties
\item klassieke computer is bijzonder geval van quantumcomputer
\item klassieke poorten EN OF en NOT
\item schema lezen met klassieke poorten
\item statemachines gebruiken om veranderingnen in de product state te kunnen analyseren
\end{itemize}
\end{document}