\documentclass[../main.tex]{subfiles}
\begin{document}
\onlyinsubfile{
\setcounter{chapter}{0}
}
\notinsubfile{}
\section{werkblad Interfaces}\label{sec:interfaces}

\marginpar{\hfill\fbox{
\begin{minipage}[t]
{0.9\marginparwidth}naam:\hfill\vspace{1cm}
\end{minipage}
}}
\marginpar{\hfill\fbox{
\begin{minipage}[t]
{0.9\marginparwidth}klas:\hfill\vspace{1cm}
\end{minipage}
}}
\marginpar{\hfill\fbox{
\begin{minipage}[t]
{0.9\marginparwidth}datum:\hfill\vspace{1cm}
\end{minipage}
}}
Er zijn een aantal manieren om quantumalgoritmen uit te voeren of the simuleren. Dit werkblad behandelt er enkele. een voorbehoud hier: de interfaces veranderen voortdurend. Er zijn vele platforms en quantum programmeertalen. 
\begin{itemize}
\item Quantum Inspire (Qutech, Delft)
\item Quirk (Google)
\item Qiskit (IBM) interface en python notebooks packages

\end{itemize}
en de pythonnotebooks omgevingen:
\begin{itemize}[resume]
\item cirq (google)
\item qiskit (IBM
\end{itemize}

In dit werkblad alleen aanwijzingen om aan de slag te gaan. 

-Profielwerkstukken

\end{document}
