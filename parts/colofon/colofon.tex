\documentclass[../../main.tex]{subfiles}
\begin{document}
%% my chapter 1 content
%\onlyinsubfile{
%\setcounter{chapter}{-1}
%this only appears if chapter00.tex is compiled (not when main.tex is compiled)}
%\notinsubfile{
%\setcounter{chapter}{-1}
%this only appears if main.tex is compiled (not when chapter00.tex is compiled)}
%% more of my chapter 1 content
%% 
\section*{Colofon}

De lesmodule \textit{Kansen met Quantum} is geschreven in opdracht van Vereniging NLT en is bestemd voor de lessen NLT (Natuur, Leven en Technologie). De module is op 15~oktober~2022  gecertificeerd door de Vereniging NLT, onder nummer 6140-099-E1-2F1 en is geldig tot 16~oktober~2027. De module is relevant voor de domeinen:
 
\marginpar{\vspace{0cm}
\includegraphics[width=0.95\marginparwidth]{./img/nltgecertificeerd.jpg}}

\begin{tabular}{lll}
VWO & E1:& Methoden en technieken van technologische\\
    &    & ontwikkeling \\ 
    & F1:& Fundamentele theorie\"en \\ 
\end{tabular} 

\begin{tabular}{lll}
versie~1.0: & oktober 2022 & gecertificeerd\\
versie~1.1: & oktober 2023 & update\\
\end{tabular} 

De module is ontwikkeld door:
\begin{itemize}[nosep]
\item \textbf{Guido Linssen} (v/h Gymnasium Feliseum, Velsen)
\item \textbf{Annemarije Zwerver} (QuTech, Delft)
\item \textbf{Martin Mollema} (Scala College, Alphen aan de Rijn)
\item \textbf{Henk Buisman} (eindredactie, Universiteit Leiden)
\end{itemize}
Verder werkten mee:
\begin{itemize}[nosep]
\item \textbf{Hans van Bemmel}(didactisch advies)
\item \textbf{Ronald de Wolf} (wetenschappelijk advies)
\item \textbf{Lianne van der Meer} (redactie)
\item \textbf{Dennis Wijmer} illustraties
\end{itemize}

\textbf{Versie info}

Versie 1.0 (oktober 2022) NLT-Gecertificerd.\\
Versie 1.1 (oktober 2023) Talloze typo's verbeterd. H2: Het meetprobleem heeft een eigen paragraaf. Het Bell experiment is herschreven. Docentenhandleiding bijgewerkt.

\marginpar{\vspace{0cm}
\includegraphics[width=0.95\marginparwidth]{./img/88x31.png}}
\textbf{Copyright}: Deze module is beschikbaar onder Creative Commons licentie \hrefqr[-4.5cm]{https://creativecommons.org/licenses/by-nc/4.0/}{CC~BY-NC~4.0}. Alle rechten voorbehouden.

Het gecertificeerde materiaal (voor docenten en leerlingen) vindt u op  de moduledatabase van de Vereniging NLT.
De geactualiseerde versie van de leerlingenhandleiding en aanvullend materiaal staan op de \hrefqr[-1cm]{http://www.quantumrules.nl/}{website van het project} Quantum Rules. 

Docenten mogen deze module aanpassen voor gebruik in de les, zonder daarbij de certificering teniet te doen, onder voorwaarde dat in het colofon vermeldt staat dat het een aangepaste versie betreft.

Deze module is tot stand gekomen onder verantwoordelijkheid van de Vereniging NLT met medefinanciering van de Universiteit Leiden.
\end{document}