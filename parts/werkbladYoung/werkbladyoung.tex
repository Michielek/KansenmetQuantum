\documentclass[../../main.tex]{subfiles}
\begin{document}
\onlyinsubfile{
\setcounter{chapter}{0}
}
\notinsubfile{}
\section{Proef van Young 1}\label{sec:wbYoung1}

\marginpar{\hfill\fbox{
\begin{minipage}[t]
{0.9\marginparwidth}naam:\hfill\vspace{1cm}
\end{minipage}
}}
\marginpar{\hfill\fbox{
\begin{minipage}[t]
{0.9\marginparwidth}klas:\hfill\vspace{1cm}
\end{minipage}
}}
\marginpar{\hfill\fbox{
\begin{minipage}[t]
{0.9\marginparwidth}datum:\hfill\vspace{1cm}
\end{minipage}
}}
De opstelling wordt gebouwd op een T-vormig plankje. Aan de achterkant wordt een rode laserpen gemonteerd. Aan de voorkant kan een visitekaartje in een sleuf heen en weer geschoven worden. In het visitekaartje zitten drie spleetpatronen. De spleten hebben telkens dezelfde breedte (\SI{0.1}{\milli\meter}), maar staan op verschillende afstand. Het linker patroon is een enkele spleet, in het midden een dubbelspleet (hartafstand \SI{0.3}{\milli\meter}). Het rechter patroon heeft een heel kleine afstand tussen de spleten. We gebruiken het rechter patroon niet in deze experimenten. De laser geeft voldoende licht om het experiment in een enigszins verduisterde ruimte uit te voeren.
\begin{enumerate}
\item Schuif de visitekaart in de gleuf en belicht de enkelspleet. Het enkelspleetpatroon is zo'n \SI{10}{\centi\meter} breed op \SI{3}{\meter} afstand. Hang dit werkblad aan de muur en trek het intensiteitspatroon over. Er staat een hulplijntje voor de symmetrieas van het patroon.
\item Meet de afstand tussen muur en spleetpatroon en noteer die in het vakje hiernaast. Verander de afstand niet meer.
\item Belicht de dubbelspleet. Controleer of beide spleten evenveel licht doorlaten door een papiertje vlak achter de spleten te houden. Projecteer het dubbelspleetpatroon in het midden van het volgende vak van het werkblad. Trek het intensiteitspatroon weer over.
\end{enumerate}

\marginpar{\vspace{-4.0cm}\hfill\fbox{
\begin{minipage}[t]
{0.9\marginparwidth}afstand tussen muur en spleet:\hfill\vspace{1cm}
\end{minipage}
}}

\noindent\begin{tikzpicture}
\node [extra] (box){%
\begin{minipage}{\dimexpr\linewidth-2\fboxrule-2\fboxsep\relax}
\hspace{2.2in}\rule{.1pt}{30pt}
\hfill
%\vspace{0.5in}
\end{minipage}
};
\node[extratitle] at (box.north west) {enkelspleet};
\end{tikzpicture}%
\vspace{0.1in}
\noindent\begin{tikzpicture}
\node [extra] (box){%
\begin{minipage}{\dimexpr\linewidth-2\fboxrule-2\fboxsep\relax}
\hspace{2.2in}\rule{.1pt}{30pt}
\hfill
%\vspace{0.5in}
\end{minipage}
};
\node[extratitle] at (box.north west) {dubbelspleet};
\end{tikzpicture}%

We hebben het enkel- en dubbelspleetexperiment uitgevoerd met telkens dezelfde spleetbreedte (\SI{0.1}{\milli\meter}). Om experimenten te vergelijken moet je beseffen welke onderdelen hetzelfde zijn gebleven en welke zijn veranderd. 
\begin{enumerate}[resume]
\item Wat zijn de overeenkomsten en verschillen in de opzet van de twee experimenten? Hoe vind je die terug in de resultaten, de patronen op de muur? 
\end{enumerate}
\iffalse
\begin{center}
\begin{tabular}{|l|l|l|}
\hline
\multicolumn{3}{|c|}{Enkel- en dubbelspleet} \\ \hline
              & overeenkomst   & verschil   \\ \hline
ontwerp       &                &            \\ \hline
waarneming    &                &            \\ \hline
\end{tabular}
\end{center}
\fi
\notepadlines[6]

\begin{enumerate}[resume]
\item Wat gebeurt er met het dubbelspleetpatroon als je \'e\'en van de spleten afdekt?
\end{enumerate}

\notepadlines[4]
\vspace{0.125in}

\begin{enumerate}[resume]
\item Waar op de muur moet ik twee lichtsensors plaatsen waarmee ik kan waarnemen of ik met een enkel- of met een dubbel\-spleetexperiment te maken hebt?
\end{enumerate}

\notepadlines[4]
\vspace{0.125in}

\begin{enumerate}[resume]
\item Teken in de intensiteitspatronen van het enkel- en dubbel\-spleetexperiment honderd (nu ja) losse fotonen die zouden bijdragen aan de opbouw van het patroon. Waar komen de meeste fotonen terecht, waar komen ze zeker niet?
\end{enumerate}
\end{document}
