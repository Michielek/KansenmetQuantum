\documentclass[../../main.tex]{subfiles}
\begin{document}
\onlyinsubfile{
\setcounter{chapter}{0}
}
\notinsubfile{}
\section{Proef van Young 2}\label{sec:wbYoung2}

\marginpar{\hfill\fbox{
\begin{minipage}[t]
{0.9\marginparwidth}naam:\hfill\vspace{1cm}
\end{minipage}
}}
\marginpar{\hfill\fbox{
\begin{minipage}[t]
{0.9\marginparwidth}klas:\hfill\vspace{1cm}
\end{minipage}
}}
\marginpar{\hfill\fbox{
\begin{minipage}[t]
{0.9\marginparwidth}datum:\hfill\vspace{1cm}
\end{minipage}
}}
We gebruiken weer de opstelling met het T-vormige plankje waarop de laser gemonteerd is. Licht van laserpennen is enigszins gepolariseerd. Bij de opstelling is een schuifje met twee loodrecht gemonteerde polarisatoren. Zet een van de filters in de bundel en draai de laser zo ver rond dat de afgebeelde lichtvlek het donkerst is. Draai de laser nu \SI{45}{\degree}. Controleer dat beide filters in het schuifje evenveel licht doorlaten. De laser staat nu goed afgesteld. Markeer de bovenkant van de laser.

\begin{enumerate}
\item Leg het schuifje even opzij en stel de dubbelspleet in zoals je in experiment~\ref{sec:wbYoung1} hebt gedaan. Als het goed is kun je het patroon van de dubbelspleet reproduceren.
\item Schuif het plaatje met de twee loodrechte polarisatieplaatjes n\'a de dubbelspleet. De fotonen gaan dus eerst door de dubbelspleet, en daarna door het filter. Het schuifje kun je heen en weer bewegen terwijl het visitekaartje op zijn plaats blijft. Dat wordt op zijn plaats gehouden door een stukje rubber. Plaats het schuifje z\'o, dat elk van de filters \'e\'en spleet afdekt.
\item Hang dit werkblad op dezelfde afstand als in experiment~\ref{sec:wbYoung1}. Teken het intensiteitspatroon hieronder.
\end{enumerate}
\noindent\begin{tikzpicture}
\node [extra] (box){%
\begin{minipage}{\dimexpr\linewidth-2\fboxrule-2\fboxsep\relax}
\hspace{2.2in}\rule{.1pt}{30pt}
\hfill
%\vspace{0.5in}
\end{minipage}
};
\node[extratitle] at (box.north west) {Dubbelspleet met gekruiste polarisatoren};
\end{tikzpicture}%

Het patroon komt overeen met een enkelspleet patroon. Kun je deze waarneming verklaren met klassieke theorie (intensiteit van de bundels) en vanuit het gedrag van enkele fotonen?
\vspace{0.125in}
\notepadlines[4]

\begin{enumerate}[resume]
\item Neem nu een los vel polaroid en hou dat ergens tussen het apparaat en de afbeelding. Draai het rond. Bij \SI{45}{\degree} zie je het dubbelspleetpatroon herstellen. Trek het patroon over op het werkblad.
\end{enumerate}

\vspace{0.1in}
\noindent\begin{tikzpicture}
\node [extra] (box){%
\begin{minipage}{\dimexpr\linewidth-2\fboxrule-2\fboxsep\relax}
\hspace{2.2in}\rule{.1pt}{30pt}
\hfill
%\vspace{0.5in}
\end{minipage}
};
\node[extratitle] at (box.north west) {met extra polarisator op \SI{45}{\degree}};
\end{tikzpicture}%
%\vspace{0.3in}


\begin{enumerate}[resume]
\item Kun je dit verklaren met klassieke theorie (intensiteit van doorgelaten bundels)?
\end{enumerate}
\notepadlines[5]
\vspace{0.25in}
 
\begin{enumerate}[resume]
\item \label{itm:last}Neem de stappen van het experiment door alsof het met \'e\'en foton tegelijk wordt uitgevoerd. Waar komt de interferentie dan vandaan?
\end{enumerate}
\notepadlines[5]
\end{document}
